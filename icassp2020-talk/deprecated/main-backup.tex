\documentclass[10pt]{beamer}
\usetheme{yeast}

\usepackage{lipsum}
\usepackage{tabularx}
\renewcommand{\arraystretch}{2}
\renewcommand\tabularxcolumn[1]{m{#1}}

\title{Meta Learning for \\ End-to-End Low-Resource Speech Recognition}
\subtitle{Speech Processing \& Machine Learning Laboratory}
\author{Jui-Yang Hsu}
\date{\today}

\begin{document}

\maketitle

\maketoc

\section{Basic Elements}

\begin{frame}{Emphasized Text}
  Text can have different \textbf{weight}.
  And not only weight, it could also \textit{be italic}.

  But most of the time, simply use \texttt{\textbackslash{}emph\{\}} could be the best choice.
  In normal text, text being emphasized looks exactly \emph{like italic text}.\footnote{But it seems that this is not working in italic mode.}

  \textit{Sometimes you really need to emphasize something, you might want it not only to be italic, but also \textbf{be bold}.}

  Other than italic and bold text, text could \alert{be colored} with \texttt{\textbackslash{}alerted\{\}}.
\end{frame}

\begin{frame}{Ordered and Unordered Lists}
  The ordered list looks like this:
  \begin{enumerate}
    \item The first item
    \item second one
          \begin{enumerate}
            \item the nested first item
            \item the second one
                  \begin{enumerate}
                    \item the most indented one
                    \item And the last one
                  \end{enumerate}
            \item No this is the last one
          \end{enumerate}
  \end{enumerate}

  And the unordered one looks like this:
  \begin{itemize}
    \item The first item
    \item and the second one
          \begin{itemize}
            \item The first nested item
            \item the second one
                  \begin{itemize}
                    \item Foo
                    \item bar
                  \end{itemize}
          \end{itemize}
  \end{itemize}
\end{frame}

\begin{frame}{Figure}
  \begin{figure}
    \centering
    %\includegraphics[width=.7\linewidth]{images/drew-coffman-Azli_kcxRNE-unsplash.jpg}
    \caption{Photo by Drew Coffman on Unsplash}
    \label{fig:breads-by-drew-coffman}
  \end{figure}
\end{frame}

\begin{frame}{Table}
  In my opinion, \texttt{tabularx} could work better most of the time than simply using \texttt{tabular}.
  \begin{table}
    \centering
    \begin{tabularx}{\textwidth}{|c|X|X|}
      \hline
      \textbf{Characteristics} & \textbf{Mold} & \textbf{Yeast} \\ \hline\hline
      Appearance
      & Fuzzy appearance and can be orange, green, black, brown, pink or purple in color
      & White and thready \\ \hline
      Uses
      & Useful in biodegradation, food production (cheese)
      & Makeing of alcoholic beverages, used in baking, and industrial ethanol production \\ \hline
    \end{tabularx}
    \caption{Molds v.s. Yeasts}
    \label{tab:molds-vs-yeasts}
  \end{table}
\end{frame}

\subsection{Elements Good for Presentation}

\begin{frame}{Blocks}
  Blocks are used to highlight some text.
  \begin{block}{Block}
    Just a block.
  \end{block}
  \begin{alertblock}{Alerted Block}
    This is an alerted block.
  \end{alertblock}
  \begin{exampleblock}{Example Block}
    And this is an example block.
  \end{exampleblock}
\end{frame}

\subsection{Overlay Animation}

\begin{frame}{Animated}
  \begin{itemize}
    \item <1-> This first item
    \item <1-> The second item
    \item <2-> The third item is hidden at first
  \end{itemize}
\end{frame}

\section{Math Equations}

\begin{frame}{Display and Inline Mode}
  Many claim that the most beautiful equation is Euler's equation.
  \[ e^{\pi i} = 1 \]

  Long ago, Johann Bernoulli noted that
  $$ \frac{1}{1+x^2} = \frac{1}{2}\left( \frac{1}{1-ix} + \frac{1}{1+ix} \right) $$

  And Roger Cotes in 1714 discovered that $ ix = \ln(\cos x + i \sin x) $
\end{frame}

\subsection{Baum-Welch Algorithm}

\begin{frame}{Forward Procedure}
  \textit{Forward algorithm}: define a forward variable $\alpha_t(i)$
  \begin{align}
    \alpha_t (i)
     & = P(o_1, o_2, \dots, o_t,\ q_t = i\ |\ \lambda)                                                              \\
     & = \text{Prob}\,[\,\text{observing } o_1, o_2, \dots, o_t, \text{ state } i \text{ at time } t\ |\ \lambda\,]
  \end{align}

  \begin{description}
    \item[Initialization]
          \begin{equation}
            \alpha_1(i) = \pi_i b_i (o_1),\ 1 \leq i \leq N
          \end{equation}
    \item[Induction]
          \begin{multline}
            \alpha_{t+1}(j) = \left[\ \sum_{i=1}^{N} \alpha_t(i) a_{i j}\ \right] \cdot b_j(o_{t+1}),\\
            1 \leq t \leq T-1,\ 1 \leq j \leq N
          \end{multline}
    \item[Termination]
          \begin{equation}
            P\left( \bar{O}\ |\ \lambda \right) = \sum_{i=1}^{N} \alpha_T(i)
          \end{equation}
  \end{description}
\end{frame}

\begin{frame}{Backward Procedure}
  \textit{Backward algorithm}: define a backward variable $\beta_t(i)$
  \begin{align}
    \beta_t(i)
     & = P(o_{t+1}, o_{t+2}, \dots, o_T\ |\ q_t = i, \lambda)                                                               \\
     & = \text{Prob}\,[\,\text{observing } o_{t+1}, o_{t+2}, \dots, o_T\ | \text{ state } i \text{ at time } t,\ \lambda\,]
  \end{align}

  \begin{description}
    \item[Initialization]
          \begin{equation}
            \beta_T(i) = 1,\ 1 \leq i \leq N
          \end{equation}
    \item[Induction]
          \begin{multline}
            \beta_{t}(i) = \sum_{j=1}^{N} a_{i j}\ b_j (o_{t+1})\  \beta_{t+1}(j),\\
            t = \{ T-1, T-2, \dots, 1\},\ 1 \leq i \leq N
          \end{multline}
  \end{description}
\end{frame}

\section{And This Is Simply a Test to See Whether a Very Long Section Name Looks Good in the Footline}

\begin{frame}{Lipsum}
  \begin{quotation}
    \lipsum[1]
  \end{quotation}
\end{frame}

\end{document}
