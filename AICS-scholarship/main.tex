\documentclass[12pt,UTF8,fntef]{article}
\usepackage[left=1.8cm, right=1.8cm, top=1.8cm, bottom=1.8cm]{geometry}
%\usepackage[utf8]{inputenc}
\usepackage[english]{babel}
\usepackage{graphicx}
\usepackage[unicode, pdfborder={0 0 0}, bookmarksdepth=-1]{hyperref}
\usepackage[usenames, dvipsnames]{color}
\usepackage[shortlabels, inline]{enumitem}
\usepackage{fancyhdr}
\usepackage{amssymb}
\usepackage{amsmath}
\usepackage{hyperref}
\usepackage{float}
\usepackage{cite}
\usepackage{siunitx}
\usepackage{enumitem}

\hypersetup{
    pdftoolbar=true,        % show Acrobat’s toolbar?
    pdfmenubar=true,        % show Acrobat’s menu?
    pdffitwindow=false,     % window fit to page when opened
    pdfstartview={FitH},    % fits the width of the page to the window
    %pdftitle={前瞻語音計畫申請},    % title
    pdfauthor={Jui-Yang Hsu},     % author
    pdfnewwindow=true,      % links in new PDF window
    colorlinks=true,       % false: boxed links; true: colored links
    linkcolor=purple,          % color of internal links (change box color with linkbordercolor)
    citecolor=blue,        % color of links to bibliography
    filecolor=magenta,      % color of file links
    urlcolor=cyan           % color of external links
}

%\usepackage[square,numbers]{natbib}
%\bibliographystyle{abbrvnat}
\renewcommand{\baselinestretch}{2}
\usepackage{xeCJK}
\usepackage{fontspec}
\setCJKmainfont[BoldFont=Noto Serif CJK TC Bold]{Noto Serif CJK TC}
\title{AICS 博士生獎學金推薦函}
\author{徐瑞陽 (\texttt{0912585820})}
%\email{r07921053@ntu.edu.tw}
\date{}

\begin{document}

\maketitle

  瑞陽於大三上開始在語音實驗室進行專題研究,當時他與另一位專題生合作,藉由閱讀文獻及修習相關的研究所課程,開始接觸語言及語音的前沿知識,初探研究的奧妙。在對學理有了基本認識後,他們於三下參與了國際語義評測大會 (SemEval 2016) 的線上競賽,學習如何將所學用於解決真實世界的問題。透由這次比賽,除了對機器學習的演算法有了更深一層的了解,也實做了當時最新提出的端對端訓練模型,更於賽後發想了將深度學習與文法結構結合的框架,並將此想法應用於當時實驗室正在進行的語音問答系統研究中。該研究的目標是讓機器可以聽一段聲音 (例如: 上課的錄音),便能回答相關問題,他們所提出的方法在托福聽力測驗上達到了最好的結果,並將結果寫成論文,投稿至頂尖的語音國際會議 SLT 2016,也成功地被接受,前往美國發表論文。該論文於會場也吸引了相當多對此技術有興趣的團隊,其中包含了蘋果公司自然語言團隊的科學家,在與他們更深入的對談後,也邀請他們參與蘋果的暑期實習。

\vspace{1.5em}

大四時,在未來發展與家計的權衡,及徵詢了包括我在內的師長們的建議後,瑞陽決定留在語音實驗室攻讀碩士學位,但先延畢交換學生一學年 (2017/2018)。一方面先跳脫相同文化與成長環境的同溫層,與來自不同背景的學生一齊共事,拓展自己的生命經驗及視野,對碩士生涯的目標也能更趨明確;一方面語音實驗室的碩士訓練,在研究能力的培養也能讓他具備足夠的競爭力。而考量到之後赴美發展的機會,透由實驗室的人脈及自己的努力仍然不少,於是在交換時,他選擇了瑞典的皇家理工學院。在為期一年的交換期中,與來自歐陸各國的優秀同儕共事,參與了演算法競賽、業界的資料視覺化專案...等台大沒有的資源,拓寬了知識的面向;亦於連假期間規劃行程,遊歐十餘國,於閒暇時磨練自己的廚藝與文采;而在機器學習領域,則架了技術部落格,分享這兩年多來的所學,同時也梳理了知識脈絡,加深自己的理解。

\vspace{1.5em}

交換期結束後,瑞陽便立刻前往位於加州的蘋果公司總部,於自然語言團隊展開為期三個月的暑期實習。在實習期中,見識到世界頂尖的科技公司是如何去定義問題,並將最新的技術應用於現實,每天有數億人使用的產品中;也在公司快速的步調下,閱讀大量論文,找出可行的解法去解決專案中的問題,最終也在部門獲得很高的評價及正職邀請,而他所參與開發的專案也已於去年九月於 iOS 13 中亮相。

\vspace{1.5em}

於 2018 年回到台大念碩士後,有了之前的學習經驗,瑞陽已經做好了投身世界前端研究的準備。在閱讀論文時不再只是單方向的吸收,而是能去揣摩文章的絲路,並能連結其他相關的論文,串接瑣碎的知識為漸趨完整的平面,從中發現可深入的方向。在碩一的摸索後,他選擇了\textbf{元學習} (Meta Learning) 作為碩論核心,元學習的目的是希望可以讓機器自行學習「學習演算法」,也就是「學習如何學習」,是近三年被各大企業視為重點的研究項目之一,雖然具有挑戰性,但也具有相當的發展潛力,尤其將其應用於語音上的研究直到 2019 年才有了首篇論文。他於去年暑假著手開發將元學習應用於語音辨識的專案,讓機器能從數種語言中找出更好的學習語音辨識的方法,使機器在學習新的語言的語音辨識時,能在有限的標註訓練資料上,獲得更高的辨識正確率,有助於將語音辨識推廣至更多的語種及場景。而該專案因為相當模組化,目前也成為實驗室其他同學參考的框架,將元學習的技術落實於更多的語言與語音應用中。瑞陽也從中發現了可以繼續深入研究的契機,及頂尖研究讓人著迷之處,也因此決定在李宏毅老師門下繼續攻讀博士學位
。

\vspace{1.5em}

另外,瑞陽在碩一時,也承擔了實驗室的網路管理員一職,管理叢集運算資源,深入了解並維護前任網管所引進的工作排程系統,引進新的功能及設計更好的資源分配策略,過程當中不僅增進了自己對底層系統的熟稔程度,也讓同學在跑實驗時更為便利,增加大家的研究效率。


\end{document}
