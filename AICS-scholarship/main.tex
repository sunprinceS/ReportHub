\documentclass[14pt,UTF8,fntef]{memoir}
\usepackage[left=1.8cm, right=1.8cm, top=1.8cm, bottom=1.8cm]{geometry}
%\usepackage[utf8]{inputenc}
\usepackage[english]{babel}
\usepackage{graphicx}
\usepackage[unicode, pdfborder={0 0 0}, bookmarksdepth=-1]{hyperref}
\usepackage[usenames, dvipsnames]{color}
\usepackage[shortlabels, inline]{enumitem}
\usepackage{fancyhdr}
\usepackage{amssymb}
\usepackage{amsmath}
\usepackage{hyperref}
\usepackage{float}
\usepackage{cite}
\usepackage{siunitx}
\usepackage{enumitem}

\hypersetup{
    pdftoolbar=true,        % show Acrobat’s toolbar?
    pdfmenubar=true,        % show Acrobat’s menu?
    pdffitwindow=false,     % window fit to page when opened
    pdfstartview={FitH},    % fits the width of the page to the window
    %pdftitle={前瞻語音計畫申請},    % title
    pdfauthor={Jui-Yang Hsu},     % author
    pdfnewwindow=true,      % links in new PDF window
    colorlinks=true,       % false: boxed links; true: colored links
    linkcolor=purple,          % color of internal links (change box color with linkbordercolor)
    citecolor=blue,        % color of links to bibliography
    filecolor=magenta,      % color of file links
    urlcolor=cyan           % color of external links
}

%\usepackage[square,numbers]{natbib}
%\bibliographystyle{abbrvnat}
\renewcommand{\baselinestretch}{2}
\usepackage{xeCJK}
\usepackage{fontspec}
\setCJKmainfont[BoldFont=Noto Serif CJK TC Bold]{Noto Serif CJK TC}
\title{AICS 博士生獎學金推薦函}
\author{徐瑞陽 (\texttt{0912585820})}
%\email{r07921053@ntu.edu.tw}
\date{}

\begin{document}

%\maketitle
\noindent 敬啟者:

茲以本函鄭重推薦台大電機所計算機科學組碩士班二年級學生徐瑞陽同學申請貴單位 AICS 博士生計畫。徐同學的碩士研究是在本人和李宏毅教授的語音實驗室中進行,由李宏毅教授指導;但他早在大三就進了實驗室,在本人的指導下,進行大學部學生的專題研究並有突出成績,故本人對他十分瞭解。

\vspace{1.5em}

瑞陽大三上就與另一位大三同學開始在本人的語音實驗室中進行大學部專題研究,當時他們已修習不少研究所的課程,並在閱讀研究尖端的文獻。他們在大三下就決定參加國際語意評測大會 (SemEval 2016) 的線上競賽,嘗試學習用所學解決真實世界的問題。雖在競賽中未獲具體成績 (畢竟他們才大三),但全球性戰火的洗禮絕對是磨鍊出強將的第一步,他們不但對所學有了更深一層的體會,也在過程中實做了當時全球最新提出的模型,更於賽後發展出將深層學習 (Deep Learning) 與自然語言 (Natural Language) 的文法結構加以整合的框架,並將此想法用於當時實驗室正在進行的用托福聽力測驗來了解機器可以聽懂人類語言到何種程度的研究中。該研究是讓機器先聽一段聲音 (例如: 上課的錄音),再以與該段音訊相關的問題,來測驗機器瞭解的程度。結果他們所提出的方法出人意料地得到了非常好的成果;他們將結果寫成論文,不意外地被全球語音學界公認的會議 SLT 2016 接受,當時他們二人才唸完大三進入暑假。他們兩人一同在大四上前往美國參加該會議,並發表該論文,在會場中吸引了相當多團隊的興趣,包括蘋果公司自然語言研究團隊的科學家。在更進一步的對談後,他們兩人均獲得前往加州蘋果公司總部進行暑期實習的邀請。

  %瑞陽於大三上開始在語音實驗室進行專題研究,當時他與另一位專題生合作,藉由閱讀文獻及修習相關的研究所課程,開始接觸語言及語音的前沿知識,初探研究的奧妙。在對學理有了基本認識後,他們於三下參加了國際語義評測大會 (SemEval 2016) 的線上競賽,學習如何將所學用於解決真實世界的問題。透由這次比賽,除了對機器學習有了更深一層的了解,也實做了當時最新提出的模型,更於賽後發想了將深度學習與文法結構相結合的框架,並將此想法應用於當時實驗室正在進行的語音問答系統研究中。該研究的目標是讓機器可以聽一段聲音 (例如: 上課的錄音),便能回答相關問題,他們所提出的方法在托福聽力測驗上達到了最好的結果,並將結果寫成論文,投稿至頂尖的語音國際會議 SLT 2016,也成功地被接受,前往美國發表論文。該論文於會場也吸引了相當多對此技術有興趣的團隊,其中包含了蘋果公司自然語言團隊的科學家,在與他們更深入的對談後,也邀請他們參與蘋果的暑期實習。

\vspace{1.5em}

瑞陽大三時那位併肩作戰的戰友如今已在 MIT 攻讀博士,但瑞陽決定走一條跟大家都不一樣的路。他在未來發展與家計負擔的權衡下,決定留在語音實驗室攻讀碩士學位,並充分利用台大與全球大學合作的交換學生計畫,先延畢交換學生一學年 (2017/2018),再進碩士班。在交換學校的選擇上,他也有十分不同的思考,多數同學想的都是美國,他卻希望走和大家不同的路;考慮到歐洲的多元文化,包括歐洲語言眾多是語音語言相關研究的天堂 (不似美國只有英文及西文),歐洲又多中小型國家,情境與台灣較為接近等,也包括未來畢竟赴美發展的機會可能較多,而長時間體驗歐洲社會文化的機會顯然較少,於是他選擇了瑞典斯德哥爾摩的皇家理工學院 (KTH) ,這在歐洲是享負盛名的學院,在語音相關的研究也相當突出。他在為期一年的交換歲月中,不但有來自歐陸各國的群英聚集可以切磋共事,也充分利用了業界專案、演算法競賽等台大較為欠缺的資源;並妥為運用連續假期規劃行程,僕僕風塵走遍歐陸十餘國,掌握機會充分體驗歐洲的多元文化及社會風土,以期擴展自己的視野格局;而在專業的機器學習及語音語言領域,他則架設了技術部落格,分享兩年多來的所學,同時也梳理了知識脈絡,加深自己的理解。

%大四時,在未來發展與家計的權衡下,及徵詢了包括我在內的師長們建議後,瑞陽決定留在語音實驗室攻讀碩士學位,但先延畢交換學生一學年 (2017/2018)。一方面先跳脫相同文化與成長環境的同溫層,與來自不同背景的學生一齊共事,拓展自己的生命經驗及視野,對碩士生涯的目標也能更趨明確;一方面語音實驗室的碩士訓練,在研究能力的培養能讓他具備足夠的競爭力。而考量到之後赴美發展的機會,透由實驗室的人脈及自己的努力仍然不少,於是在交換時,他選擇了瑞典的皇家理工學院 (KTH) 。為期一年的交換歲月,與來自歐陸各國的優秀同儕共事,參與了演算法競賽、業界的資料視覺化專案...等台大沒有的資源,拓寬了知識的面向;亦於連假期間規劃行程,遊歐十餘國,於閒暇時磨練自己的廚藝與文采;而在機器學習領域,則架了技術部落格,分享這兩年多來的所學,同時也梳理了知識脈絡,加深自己的理解。

\vspace{1.5em}

在瑞典的交換期結束後,瑞陽便立刻如約前往加州蘋果公司總部,參與自然語言研究團隊為期三個月的暑期實習,見識到世界頂尖的科技公司是如何去發掘及定義問題,並將最先進的技術應用於現實生活,每天數億人使用的產品中;也無可避免地在公司快速的步調下,閱讀消化大量尖端文獻,並找出可行的方法去克服專案中的難題。他最終在該部門獲得很高的評價,包括正職邀請(可以直接在蘋果上班,無須念碩士班),而他所參與開發的專案也已於去年九月在 iOS 13 中亮相。 

%交換期結束後,瑞陽便立刻前往位於加州的蘋果公司總部,於自然語言團隊展開為期三個月的暑期實習。在實習期中,見識到世界頂尖的科技公司是如何去定義問題,並將最新的技術應用於現實,每天有數億人使用的產品中;也在公司快速的步調下,閱讀大量論文,找出可行的解法去解決專案中的問題,最終也在部門獲得很高的評價及正職邀請,而他所參與開發的專案也已於去年九月於 iOS 13 中亮相。

\vspace{1.5em}

瑞陽於 2018 年蘋果實習結束後回到台大念碩士。經歷了歐洲文化及蘋果科技的洗禮,瑞陽和當年大三開始做專題時已大不相同;他做好了投身尖端科學世界,積極探索的準備。他在閱讀文獻時已不只是單方向的吸收,而是能去體驗揣摩文獻中的思路,並觸類旁通,和其他相關的文獻、技術或問題建立連結,串接瑣碎的知識趨於更為完整的線或面,並從中發現可著力的方向。我當時告訴他,應該在碩士研究中體驗出「如何自己從讀文獻,醞釀方向到最後做出有深度、有貢獻的研究」。在碩一的摸索後,他選擇了\textbf{元學習} (Meta Learning) 作為碩論的核心,元學習的精神是希望機器可以自行學習出「學習演算法」,也就是「學習如何學習」。這是近三年來全球各前端研究團隊視為重點的研究領域之一,極具挑戰性及發展潛力,尤其在語音領域直到 2019 年才出現首篇論文,但當然也有極強勁的競爭者。他在去年暑假開始將元學習應用於語音辨識的研究,主軸是讓機器能從數種已有較多標註語料的語言中,發現更好的學習方法,使機器在學習新的語言的語音辨識技術時,能在有限的標註訓練資料上,獲得更高的辨識正確率;這將有助於將語音辨識推廣至更多的語言種類及應用情境中。他的程式架構因為相當模組化,目前也常為實驗室中其他同學所參考使用,以期將元學習的技術落實於更多語言語音的研究環境中。瑞陽自己也在這中間發現了繼續深入研究的契機,及自己從事尖端研究的潛力,並體驗到世界前端的頂尖研究讓人振奮及著迷之處。他因而決定在宏毅教授指導下,繼續攻讀博士學位。

%於 2018 年回到台大念碩士後,有了之前的學習經驗,瑞陽已經做好了投身世界前端研究的準備。在閱讀論文時不再只是單方向的吸收,而是能去揣摩文章的思路,並能連結其他相關的論文,串接瑣碎的知識為漸趨完整的平面,從中發現可深入的方向。在碩一的摸索後,他選擇了\textbf{元學習} (Meta Learning) 作為碩論核心,元學習的目的是希望可以讓機器自行學習「學習演算法」,也就是「學習如何學習」,是近三年來被各大企業視為重點的研究項目之一,雖然具有挑戰性,但也有相當的發展潛力,尤其將其應用於語音上的研究直到 2019 年才有了首篇論文。他於去年暑假著手開發將元學習應用於語音辨識的專案,讓機器能從數種語言中找出更好的學習方法,使機器在學習新的語言的語音辨識時,能在有限的標註訓練資料上,獲得更高的辨識正確率,有助於將語音辨識推廣至更多的語種及場景。而該專案因為相當模組化,目前也成為實驗室其他同學參考的框架,將元學習的技術落實於更多的語言與語音應用中。瑞陽也從中發現了可以繼續深入研究的契機,及頂尖研究讓人著迷之處,也因此決定在李宏毅老師門下繼續攻讀博士學位

\vspace{1.5em}

此外,值得一提的是,瑞陽在碩一時,曾擔任了本人及宏毅教授實驗室的網管一年。我們實驗室相信是全台大電機系機器規模最龐大,系統最複雜,研究內容最多元的實驗室,為了妥善管理叢集運算資源,他不但深入了解及維護前任網管所引進的工作排程系統,並引進不少新的功能及設計更好的資源分配策略,不僅增加了他自己對底層系統的熟稔度,也學到如何讓數十個同學在公平獲得計算資源的前提下各取所需,及在系統中實現所制定的策略(尤其在接近論文截止期限時更顯重要),讓大家的研究效率可以最佳化。他在這方面可說是有相當好的成果。

%另外,瑞陽在碩一時,也承擔了實驗室的網路管理員一職,管理叢集運算資源,深入了解並維護前任網管所引進的工作排程系統,引進新的功能及設計更好的資源分配策略,過程當中不僅增進了自己對底層系統的熟稔程度,也讓同學在跑實驗時更為便利,增加大家的研究效率。

\vspace{1.5em}

基於上述,本人認為徐瑞陽同學十分適合貴單位的 AICS 博士生計畫,特予極力推薦。

\begin{flushright}
台大電機系教授
\vspace{2.5em}

敬上
\end{flushright}


\end{document}
