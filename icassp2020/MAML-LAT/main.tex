% Template for ICASSP-2020 paper; to be used with:
%          spconf.sty  - ICASSP/ICIP LaTeX style file, and
%          IEEEbib.bst - IEEE bibliography style file.
% --------------------------------------------------------------------------
%\usepackage{multicol}
%\usepackage{multirow}
\documentclass{article}
\usepackage{color}
\usepackage{booktabs}
\usepackage{amsfonts}
\usepackage{spconf,amsmath,graphicx}
\usepackage{siunitx}
\usepackage{tikz}
\usepackage{pgfplotstable}
\usepackage{subfig}
\usepackage{float}
%\usepackage[preprint]{spconf}
\usepackage[OT1]{fontenc} % TODO: 之後放到 overleaf 要移掉!!

% Example definitions.
% --------------------
\def\x{{\mathbf x}}
\def\L{{\cal L}}
\pgfplotsset{compat=1.16}

% ------
\title{Meta Learning for End-to-End Low-Resource Speech Recognition}
%
% Single address.
% ---------------
%\name{Jui-Yang Hsu, Yuan-Jui Chen,  Hung-yi Lee}
\name{Jui-Yang Hsu\qquad Yuan-Jui Chen\qquad  Hung-yi Lee}
\address{National Taiwan University \\
\small{\texttt{\{r07921053, r079xxxxx, hungyilee\}@ntu.edu.tw}}}
%
% For example:
% ------------
%\address{School\\
%	Department\\
%	Address}
%
% Two addresses (uncomment and modify for two-address case).
% ----------------------------------------------------------
%\twoauthors
%  {A. Author-one, B. Author-two\sthanks{Thanks to XYZ agency for funding.}}
%	{School A-B\\
%	Department A-B\\
%	Address A-B}
%  {C. Author-three, D. Author-four\sthanks{The fourth author performed the work
%	while at ...}}
%	{School C-D\\
%	Department C-D\\
%	Address C-D}
%
\begin{document}
%\ninept
%
\maketitle
%
\begin{abstract}
  %\textcolor{red}{(TBD BEGIN)} With the recent advances of deep learning, integrating the main modules of automatic speech recognition (ASR) such as acoustic model, pronunciation lexicon and language model into a single end-to-end model is highly attractive. Connectionist Temporal Classification (CTC) lends itself on such end-to-end approach by introducing an additional blank symbol and specifically-designed loss function optimizing to generate the correct character sequences from the speech signal directly, without framewise phoneme alignment in advance. With many recent results, end-to-end deep learning has created larger interest in speech community. \textcolor{red}{(TBD END)}
\end{abstract}
%
\begin{keywords}
  meta-learning, low-resource, multi-lingual speech recognition, language adaptation, IARPA-BABEL
\end{keywords}
%
\section{Introduction}
With the recent advances of deep learning, integrating the main modules of automatic speech recognition (ASR) such as acoustic model, pronunciation lexicon and language model into an single end-to-end model is highly attractive. Connectionist Temporal Classification (CTC) \cite{graves2006connectionist} lends itself on such end-to-end approach by introducing an additional blank symbol and specifically-designed loss function optimizing to generate the correct character sequences from the speech signal directly, without framewise phoneme alignment in advance. With many recent results \cite{hannun2014deep, amodei2016deep, collobert2016wav2letter}, end-to-end deep learning has created larger interest in speech community.

However, such end-to-end ASR system requires a huge amount of paired speech-transcription data, which is costly. For most languages in the world, they are lacking of sufficient paired data for training. Pretraining on other language sources as the initialization, then fine-tuning on target language is the dominant approach in such low-resource setting, also known as multilingual transfer learning / pretraining (MultiASR) \cite{vu2014multilingual, tong2017investigation}. The backbone of MultiASR is a multitask model with shared hidden layers (encoder), and many language-specific heads. The idea of such model structure is to learn an encoder to extract language-independent representations to build better acoustic model from many sources of languages. The success of ``language-independent'' features to improve ASR performance compared to monolingual training has been showed in many recent works \cite{cho2018multilingual, dalmia2018sequence}.

\begin{figure}[t]
  \centering
  \includegraphics[width=\linewidth]{figs/meta_idea.png}
%  \vspace{2.0cm}
  \label{fig:meta-idea}
  %TODO: Need to be replaced
  \caption{Illustration of the difference between Multitask learning and Meta Learning (\textcolor{red}{WILL BE REPLACED})}

\vspace{-20pt}
\end{figure}

%TODO: sounds weird
With its success, various variants of MultiASR have been proposed. Langauge-adversarial training approaches \cite{Yi2018AdversarialMT, adams2019massively} introduce language-adversarial classification objective to the shared encoder, negating the gradients backproped from the language classifier to encourage the encoder to extract more language-independent representations. Hierarchical approaches \cite{Sanabria2018HierarchicalMT} introduce different granularity objective through combining both character and phoneme prediction at different levels of the model.

In this paper, we provide a novel research direction following up on the idea of multilingual pretraining, \textbf{Meta learning}. Without introducing additional modules like adversarial training, or requiring phoneme level annotation (usually through lexicon) like hierarchical approaches. We only need to modify the optimization process following meta learning training scheme.

Meta learning, or learning-to-learn, has recently received huge interests in machine learning community. The goal of meta learning is to solve the problem of ``fast adaptation on unseen data'', which is aligned with our low-resource setting. With its success in computer vision under few-shot setting \cite{rusu2018meta, snell2017prototypical, vinyals2016matching}, there have been some works in language and speech processing being proposed, for instance, language transfer in neural machine translation \cite{gu2018meta}, dialogue generation \cite{mi2019meta}, and speaker adaptative training \cite{klejch2018learning}.

\textcolor{red}{(TBD BEGIN)} some observation and improvement in our paper \textcolor{red}{(TBD END)}

\label{sec:intro}


\section{Proposed Approach}
\label{sec:approach}

\subsection{Multilingual CTC Model}
\begin{figure}[t]
    \centering
    \includegraphics[width=\linewidth]{figs/model_arch.png}
    \caption{Multilingual CTC Model Architecture \textcolor{red}{WILL BE REPLACED}}
    \label{fig:model-arch}
    %\vspace{-10pt}
\end{figure}

We used the model architecture as illustrated in Fig.~\ref{fig:model-arch}, the shared encoder is parameterized by $\theta$, and the set of language-specific heads are parameterized by $\theta_h$ ($\theta_{h,l}$ means the head for $l$-th language). Let the dataset be $D$, composed of paired data $(X,C)$. Let $X = x_1, x_2, \cdots, x_T$ with length $T$ as input feature, $C = c_1, c_2, \cdots, c_L$ with length $L$ as target label. $X$ is encoded into sequence of hidden states $H = h_1, h_2, \cdots, h_{L^\prime}$ with length $L^\prime$ through the shared encoder, then fed into the language-specific head of the corresponding language with softmax activation to output the prediction sequence $\hat{C} = \hat{c_1}, \hat{c_2}, \cdots, \hat{c_{L^\prime}}$.
%\vspace{-2pt}

\textbf{CTC Loss}. CTC computes the posterior probability as below,

\begin{equation}
  P(C|X) = \sum_{\pi \in \mathcal{Z}(C)} P(\pi|X)
\end{equation}
where $\pi$ is the repeated character sequence  of $C$ with additional blank label, and $\mathcal{Z}(C)$ is the set of all possible sequences $\pi$ given sequence $C$. For each $\pi$, we can approximate the posterior probability as below,

\begin{equation}
  P(\pi|X) \approx \prod_{i=1}^{L^\prime} P(\hat{c_i}|X)
\end{equation}

The loss function of the model on $D$ is then defined as:

\begin{equation}
  \label{eq:ctc-loss}
  \mathcal{L}_D(\theta, \theta_h) = - \log P(C|X)
\end{equation}

\subsection{Meta Learning for Low-Resource ASR}

The idea of MAML is to learn initialization parameters from a set of tasks. 
In the context of ASR, we can view different languages as different tasks.
Given a set of source tasks $\mathcal{D}=\lbrace D_1, D_2, \cdots, D_K \rbrace$, MAML learns from $\mathcal{D}$ to obtain good initialization parameters $\theta^{\star}$ for the shared encoder.
$\theta^{\star}$ yields fast task-specific learning (fine-tuning) on target task $D_t$ and obtains $\theta^{\star}_t$ and $\theta^{\star}_{h,t}$ (the parameters obtained after fine-tuning on $D_t$). 
MAML can be formulated as below, 
\begin{equation*}
  \theta^{\star}_t, \theta^{\star}_{h,t} = \texttt{Learn}(D_t;\theta^{\star}) = \texttt{Learn}(D_t;\texttt{MetaLearn}(\mathcal{D})).
\end{equation*}
The two functions,  \texttt{Learn} and \texttt{MetaLearn}, will be described in the following two subsections.
%where $\theta^{\star}_t$ is the parameter obtained after fine-tuning on $T_t$.


\subsubsection{Learn: Language-specific learning}
Given any initial parameters $\theta^0$ of the shared encoder (either random initialized or obtained from pretrained model) and the data $D_t$. The language-specific learning process is to minimize the CTC loss function defined in Eq.~\ref{eq:ctc-loss}.

\begin{equation}
  \label{eq:fine-tune}
  \resizebox{0.91\hsize}{!}{$
\begin{aligned}
  \theta^\prime, \theta^\prime_{h,t} = \texttt{Learn}(D_t;\theta^0) & = \arg \, \min_{\theta, \theta_{h,t}} \mathcal{L}_{D_t}(\theta, \theta_{h,t}) \\
                                                                    & = \arg \, \min_{\theta, \theta_{h,t}} \sum_{(X,C) \in D_t} -\log P(C|X)
\end{aligned}
$}
\end{equation}
\begin{table*}[ht!]
\centering
\caption{Character (\% CER)  error rate w.r.t the pretraining languages set for all 4 target languages' FLP}
\label{tab:block-results}
\begin{tabular}{@{}ccccccccc@{}}
%\begin{tabular}{l|cc|cc|cc|cc}
\toprule
Model                                    & \multicolumn{2}{c}{Vietnamese}                         & \multicolumn{2}{c}{Swahili}                        & \multicolumn{2}{c}{Tamil}                        & \multicolumn{2}{c}{Kurmanji} \\

                                         & multi           & meta                                & multi           & meta                                & multi           & meta                                & multi           & meta           \\ \midrule
\multicolumn{1}{c|}{- (no-pretrain)}         & 100.0          & \multicolumn{1}{c|}{100.0}          & 100.0          & \multicolumn{1}{c|}{100.0}          & 100.0          & \multicolumn{1}{c|}{100.0}          & 100.0          & 100.0          \\

\multicolumn{1}{c|}{Bn Tl Zu}   & 53.2          & \multicolumn{1}{c|}{36.5}          & 52.8          & \multicolumn{1}{c|}{34.4}          & 47.8          & \multicolumn{1}{c|}{34.9}          & 55.9          & 41.1          \\
\multicolumn{1}{c|}{ Tr Lt Gn} & 50.6          & \multicolumn{1}{c|}{35.1}          & 49.0          & \multicolumn{1}{c|}{32.2}          & 46.6          & \multicolumn{1}{c|}{33.2}          & 53.4          & 39.6          \\
\multicolumn{1}{c|}{Bn Tl Zu Tr Lt Gn}           & 52.3          & \multicolumn{1}{c|}{36.6}          & 51.3          & \multicolumn{1}{c|}{33.0}          & 45.8          & \multicolumn{1}{c|}{33.9}          & 54.5          & 40.2          \\ \bottomrule
%\multicolumn{1}{c|}{MLing + SWBD \& FT}       & \textbf{48.2} & \multicolumn{1}{c|}{\textbf{33.5}} & \textbf{48.7} & \multicolumn{1}{c|}{\textbf{31.9}} & \textbf{44.3} & \multicolumn{1}{c|}{\textbf{31.9}} & \textbf{51.5} & \textbf{37.8} \\ \bottomrule
\end{tabular}
\end{table*}



The learning process is optimized through gradient descent, the same as MultiASR.

\subsubsection{MetaLearn}
The initialization parameters found by MAML should not only adapt to one language well, but for as many languages as possible. In order to achieve this goal, we define the meta learning process and the corresponding meta-objective as follows.

In each meta learning episode, we sample batch of tasks from $\mathcal{D}$, then sample two subsets from each task $k$ as training and testing set, denoted as $D_{k}^{tr}$, $D_{k}^{te}$, respecitvely. First, we use $D_k^{tr}$ to simulate the language-specific learning process to obtain $\theta^\prime_k$ and $\theta^\prime_{h,k}$.


\begin{equation}
    \theta_k^\prime, \theta_{h,k}^\prime = \texttt{Learn}(D^{tr}_k; \theta)
                                         %&= \arg \, \min_{\theta, \theta_{h,k}} \mathcal{L}_{D^{te}_k}(\theta, \theta_{h,k})
\end{equation}

Then evaluate the effectiveness of the obtained parameters on $D_k^{te}$. The goal of MAML is to find $\theta$, the initialization weights of the shared encoder for fast adaptation, so the meta-objective is defined as

\begin{equation}
  \label{eq:meta-obj}
  \mathcal{L}^{meta}_{\mathcal{D}}(\theta) =  \mathbb{E}_{k \sim \mathcal{D}} \; \mathbb{E}_{D_k^{tr}, D_k^{te}} \Big [ \mathcal{L}_{D^{te}_k}(\theta^\prime_k, \theta^\prime_{h,k}) \Big ]
\end{equation}

Therefore, the meta learning process is to minimize the loss function defined in Eq.~\ref{eq:meta-obj}.

\begin{equation}
  \theta^\star = \texttt{MetaLearn}(\mathcal{D}) = \arg \min_{\theta} \mathcal{L}^{meta}_{\mathcal{D}} (\theta)
\end{equation}

We use \textit{meta gradient} obtained from Eq.~\ref{eq:meta-obj} to update the model through gradient descent (\textit{outer loop}).

\begin{equation}
  \label{eq:meta-grad}
  \theta \leftarrow \theta - \eta^\prime \sum_k \nabla_\theta \mathcal{L}_{D^{te}_k}(\theta^\prime_k, \theta^\prime_{h,k})
\end{equation}
$\eta^\prime$ is the meta learning rate. And noted that only the shared encoder is updated via Eq.~\ref{eq:meta-grad}.

MultiASR optimizes the model according to Eq.~\ref{eq:fine-tune} on all source languages directly, without considering how learning happens on the unseen language. Although the parameters found by MultiASR is good for all source languages, it may not adapt well on the target language. Unlike MultiASR, MetaASR explicitly integrates the learning process into its framework via simulating language-specific learning first, then meta-update the model. Therefore,the parameters obtained are more suitable to adapt to the unseen language. We illustrated the concept in Fig.~\ref{fig:meta-idea}, and showed in the experimental results in Section \ref{sec:results}.



\section{Experiment}
\label{sec:exp}

%\subsection{Experimental Setup}
\label{ssec:exp-setup}
In this work, we used data from the IARPA BABEL project \cite{gales2014speech}. The corpus is mainly composed of conversational telephone speech (CTS). We selected 6 languages as non-target languages for multilingual pre-training: Bengali (Bn), Tagalog (Tl), Zulu (Zu), Turkish (Tr), Lithuanian (Lt), Guarani (Gn), and 4 target languages for adaptation: Vietnamese (Vi), Swahili (Sw), Tamil (Ta), Kurmanji (Ku), and experimented different combinations of non-target languages for pretraining.

We followed the recipe provided by Espnet \cite{watanabe2018espnet} for data preprocessing and final score evaluation.  We used 80-dimensional Mel-filterbank and 3-dimensional pitch features as acoustic features. The size of the sliding window is 25ms, and the stride is 10ms. We used the shared encoder with 6-layer VGG extractor with downsampling and a 6-layer bidirectional LSTM network with 360 cells in each direction used in the previous work \cite{dalmia2018sequence}.

%\subsubsection{Pretraining} # TODO: move to approach
%\vspace{-5pt}
\begin{table*}[ht!]
\centering
\caption{Character error rate (\si{\percent} CER) w.r.t the pretraining languages set for all 4 target languages' LLP}
\label{tab:llp-table}
\begin{tabular}{@{}ccccccccc@{}}
%\begin{tabular}{l|cc|cc|cc|cc}
\toprule
Model                                    & \multicolumn{2}{c}{Vietnamese}                         & \multicolumn{2}{c}{Swahili}                        & \multicolumn{2}{c}{Tamil}                        & \multicolumn{2}{c}{Kurmanji} \\

                                         & multi           & meta                                & multi           & meta                                & multi           & meta                                & multi           & meta           \\ \midrule
\multicolumn{1}{c|}{(no-pretrain)}                   & \multicolumn{2}{c|}{71.3}                    & \multicolumn{2}{c|}{64.3}                    & \multicolumn{2}{c|}{72.0}          & \multicolumn{2}{c}{68.6}                    \\

\multicolumn{1}{l|}{Bn Tl Zu}   & 64.7          & \multicolumn{1}{c|}{58.5}          & 63.5          & \multicolumn{1}{c|}{57.3}          & 71.2          & \multicolumn{1}{c|}{77.5}          & 68.2         & 65.3          \\
\multicolumn{1}{l|}{ \qquad \qquad Tr Lt Gn} & 64.9          & \multicolumn{1}{c|}{58.0}          & 64.1          & \multicolumn{1}{c|}{60.1}          & 75.4          & \multicolumn{1}{c|}{79.0}          & 70.3          & 63.2          \\
\multicolumn{1}{l|}{Bn Tl Zu Tr Lt Gn}           & 64.7          & \multicolumn{1}{c|}{58.9}          & 63.4          & \multicolumn{1}{c|}{59.6}          & 73.4          & \multicolumn{1}{c|}{70.6}          & 66.8          & 64.6          \\ \bottomrule
%\multicolumn{1}{c|}{MLing + SWBD \& FT}       & \textbf{48.2} & \multicolumn{1}{c|}{\textbf{33.5}} & \textbf{48.7} & \multicolumn{1}{c|}{\textbf{31.9}} & \textbf{44.3} & \multicolumn{1}{c|}{\textbf{31.9}} & \textbf{51.5} & \textbf{37.8} \\ \bottomrule
\end{tabular}
\end{table*}


\begin{table*}[ht!]
\centering
\caption{Character (\% CER)  error rate w.r.t the pretraining languages set for all 4 target languages' FLP}
\label{tab:block-results}
\begin{tabular}{@{}ccccccccc@{}}
%\begin{tabular}{l|cc|cc|cc|cc}
\toprule
Model                                    & \multicolumn{2}{c}{Vietnamese}                         & \multicolumn{2}{c}{Swahili}                        & \multicolumn{2}{c}{Tamil}                        & \multicolumn{2}{c}{Kurmanji} \\

                                         & multi           & meta                                & multi           & meta                                & multi           & meta                                & multi           & meta           \\ \midrule
\multicolumn{1}{c|}{- (no-pretrain)}         & 100.0          & \multicolumn{1}{c|}{100.0}          & 100.0          & \multicolumn{1}{c|}{100.0}          & 100.0          & \multicolumn{1}{c|}{100.0}          & 100.0          & 100.0          \\

\multicolumn{1}{c|}{Bn Tl Zu}   & 53.2          & \multicolumn{1}{c|}{36.5}          & 52.8          & \multicolumn{1}{c|}{34.4}          & 47.8          & \multicolumn{1}{c|}{34.9}          & 55.9          & 41.1          \\
\multicolumn{1}{c|}{ Tr Lt Gn} & 50.6          & \multicolumn{1}{c|}{35.1}          & 49.0          & \multicolumn{1}{c|}{32.2}          & 46.6          & \multicolumn{1}{c|}{33.2}          & 53.4          & 39.6          \\
\multicolumn{1}{c|}{Bn Tl Zu Tr Lt Gn}           & 52.3          & \multicolumn{1}{c|}{36.6}          & 51.3          & \multicolumn{1}{c|}{33.0}          & 45.8          & \multicolumn{1}{c|}{33.9}          & 54.5          & 40.2          \\ \bottomrule
%\multicolumn{1}{c|}{MLing + SWBD \& FT}       & \textbf{48.2} & \multicolumn{1}{c|}{\textbf{33.5}} & \textbf{48.7} & \multicolumn{1}{c|}{\textbf{31.9}} & \textbf{44.3} & \multicolumn{1}{c|}{\textbf{31.9}} & \textbf{51.5} & \textbf{37.8} \\ \bottomrule
\end{tabular}
\end{table*}



% (TODO) Hsu:Need to discussed use one or use three?
% (TODO) Hsu: 最後一句超遨口的...叫 dev set 會不會太隨便?
\subsection{Validation Languages}
We used Limited Language Pack (LLP) which consists of 10\% Full Language Pack (FLP) of the other 3 languages to determine which pretraining step we should pick as the pretrained model for adapting on certain language's LLP and FLP. (For instance, when adapting on Vi, we use Sw, Ta, Ku s' LLP validation set as the validation set for Vi)

\subsection{Meta Learning}
For each meta learning episode, we use a single gradient step of language-specific learning with SGD when computing the meta gradient. Noted that in Eq. \ref{eq:meta-grad}, if we expand the loss term in the summation, we will find the second derivative term of $\theta$ appear. For computation efficiency, some previous works \cite{finn2017model, nichol2018reptile} showed that we could ignore the second-order term without affecting the performance too much. Therefore, we approximate Eq. \ref{eq:meta-grad} as follows.

\begin{equation}
  \theta \leftarrow \theta - \eta^\prime \sum_k \nabla_{\textcolor{red}{\theta^\prime}} \mathcal{L}_{D^\prime_k}(\theta^\prime)
\end{equation}
Also known as First-order MAML (FOMAML).

\subsection{Fine-Tuning (Adaptation) and Evaluation}
For adaptation on target language, we fine-tuned each monolingual model for 20 epochs for LLP, 18 epochs for FLP, and early-stopped on its own dev set, and evaluated on its test set. 
%We report the results in Table \ref{tab:llp-table} and \ref{tab:flp-table}.


\section{Results}
\label{sec:results}

\subsection{vs. Multitask Learning (MultiASR)}
\label{ssec:baseline-multitask}

\subsection{Training Set Size}
\label{ssec:training-size}



\section{Conclusion}
\label{sec:conclusion}
In this paper, we proposed the meta learning approach to multilingual pretraining for speech recognition. The initial experimental results showed its potential in pretraining. In future work, We plan to use more languages (from IARPA BABEL or other corpora) and different combinations for pretraining to evaluate the effectiveness of MetaASR more extensively. 

In addition, based on MAML's model-agnostic property, this approach can be applied to other network architecture like Seq2seq model, and even different applications other than speech recognition in the speech community.

% Below is an example of how to insert images. Delete the ``\vspace'' line,
% uncomment the preceding line ``\centerline...'' and replace ``imageX.ps''
% with a suitable PostScript file name.
% -------------------------------------------------------------------------


% To start a new column (but not a new page) and help balance the last-page
% column length use \vfill\pagebreak.
% -------------------------------------------------------------------------
%\vfill
%\pagebreak


%\section{REFERENCES}
% References should be produced using the bibtex program from suitable
% BiBTeX files (here: strings, refs, manuals). The IEEEbib.bst bibliography
% style file from IEEE produces unsorted bibliography list.
% -------------------------------------------------------------------------
\bibliographystyle{IEEEbib}
\newpage
\bibliography{strings,refs}

% 以下不會放到 paper ,只是畫一些圖出來而已
%\newpage
\section{Appendix}

\begin{figure}[htb]
  \centering
  %\hspace{-2.2cm}
  \begin{tikzpicture}[trim axis left, trim axis right]

  \begin{axis}[
    width=\linewidth,
    height=6.5cm,
    legend entries={MultiASR, MetaASR} ,
    xlabel = {Number of pretraining steps ($\times 1000$)},
        xmin=5,
        %xmax=130,
        grid=both,
        legend style={at={(0.,.6)},anchor=south west},
        %legend pos=inner north west,
        ylabel={CER (\si{\percent}})]
  \addplot+[smooth]table{testing/mutli3-Tamil};
  \addplot+[smooth]table{testing/meta3-tamil};
   %\addplot[style=ultra thick,dashed,] coordinates {(0,0.557) (200,0.557)};
   %\addplot[style=ultra thick,dashed, gray] coordinates {(0,0.589) (200,0.589)};
   %\addplot[style=ultra thick,dashed, brown] coordinates {(0,0.628) (200,0.628)};
  \end{axis}
  \end{tikzpicture}
  %\caption{Pretrain on EN, FI, FR, NL, RM, RU, and evaluate on }
  \caption{The learning curves of CER on Tamil's LLP (near3)}
\end{figure}

\begin{figure}[htb]
  \centering
  %\hspace{-2.2cm}
  \begin{tikzpicture}[trim axis left, trim axis right]

  \begin{axis}[
    width=\linewidth,
    height=6.5cm,
    legend entries={MultiASR, MetaASR} ,
    xlabel = {Number of pretraining steps ($\times 1000$)},
        xmin=5,
        %xmax=130,
        grid=both,
        legend style={at={(0.,.7)},anchor=south west},
        %legend pos=inner north west,
        ylabel={CER (\si{\percent}})]
  \addplot+[smooth]table{testing/mutli6-tamil};
  \addplot+[smooth]table{testing/meta6-tamil};
   %\addplot[style=ultra thick,dashed,] coordinates {(0,0.557) (200,0.557)};
   %\addplot[style=ultra thick,dashed, gray] coordinates {(0,0.589) (200,0.589)};
   %\addplot[style=ultra thick,dashed, brown] coordinates {(0,0.628) (200,0.628)};
  \end{axis}
  \end{tikzpicture}
  %\caption{Pretrain on EN, FI, FR, NL, RM, RU, and evaluate on }
  \caption{The learning curves of CER on Tamil's LLP (near6)}
\end{figure}
\end{document}
